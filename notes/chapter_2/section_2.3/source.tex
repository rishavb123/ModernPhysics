\section{Rotations and the Center of Mass}
    The \textbf{angular velocity} $\va{\omega}$ is the rate at which the angle $\theta$ changes. Its direction is determined by the right hand rule (Fingers in direction of motion, thumb is direction of $\va{\omega}$). The rate at which $\omega$ changes is the \textbf{angular acceleration} vector $\va{\alpha}$. The rotation eq uivalents for mass and force, are \textbf{rotational inertia} $I$ and \textbf{torque} $\va{\tau}$. So Newton's second law for rotation motion is
    \begin{equation*}
        \va{\tau} = I\va{\alpha} = \frac{d\va{L}}{dt}
    \end{equation*}
    where $\va{L}$ is \textbf{angular momentum}, which is defined as $$\va{L} = \va{r} \cross \va{p}$$
    Torque is similarly defined as $$\va{\tau} = \va{r} \cross \va{F}$$
    The rotational inertia $I$ is dependent on the way an objects mass is distributed. By breaking up the object of total mass $$M = \sum_i \Delta m_i$$ into many discrete portions of mass, we can set
    \begin{equation*}
        I \equiv \sum_i (\Delta m_i) r_i^2 = \int r^2 dm
    \end{equation*}
    Work one in a rotation rigid body is defined as 
    \begin{equation*}
        W = \int_{\theta_i}^{\theta_f} \tau d\theta
    \end{equation*}
    Kinetic energy of a rotating object is
    \begin{equation*}
        K = \frac{1}{2}I\omega^2
    \end{equation*}
    The \textbf{center of mass} is defined as 
    \begin{equation*}
        \va{R} = \frac{\sum_i m_i\va{r}_i}{M}
    \end{equation*}
    For an external force moving the object, the mass will act as a point mass at $\va{R}$, so 
    \begin{equation*}
        \va{F}_{\text{net, external}} = M\va{A} = M\frac{d^2\va{R}}{dt^2}
    \end{equation*}