\section{The Axiom of Completeness}
    \textbf{Axiom of Completeness.} \textit{Every nonempty set of real numbers that is bounded above has a least upper bound}
    \subsection*{Least Upper Bounds and Greatest Lower Bounds}
        \textbf{Definition} A set $A \in \textbf{R}$ is \textit{bounded above} if there exists a number $b \in \textbf{R}$ such that $a \leq b$ fro all $a \in A$. The number b is an \textit{upper bound} for $A$. 
        \newline \indent The set $A$ is \textit{bounded below} if there exists a \textit{lower bound} $l \in \textbf{R}$ so that $l \leq a$ for all $a \in A$.
        \newline \textbf{Definition} A real number $s$ is the \textit{least upper bound} for a set $A \in \textbf{R}$ if it meets two criteria:
        \newline \indent (i) $s$ is an upper bound for $A$;
        \newline \indent (ii) if $b$ is any upper bound for $A$, then $s \leq b$;
        \newline
        The least upper bound is also called the \textit{supremum} of $A$. So, $s = \text{lub } A = \text{sup } A$. 
        \newline \indent The \textit{greatest lower bound} or \textit{infimum} for $A$ is defined similarly and is denoted by inf $A$.
        \newline \indent A set can have many upper bounds, but only one least upper bound. If $s_1$ and $s_2$ are both least upper bounds, then by property (ii) we can assert $s_1 \leq s_2$ and $s_2 \leq s_1$, and that $s_1 = s_2$.
        \newline \indent A real number $a_0$ is a \textit{maximum} of set $A$ if $a_0$ is an element of $A$ and $a_0 \geq a$ for each $a \in A$. Similarly, a number $a_1$ is a \textit{minimum} of $A$ if $a_1 \in A$ and $a_1 \leq a$ for each $a \in A$.
        \newline \indent An upper bounded set is guaranteed to have a least upper bound by \textit{The Axiom of Completeness}, but it is not guaranteed to have a maximum. A supremum can exist and not be a maximum (if the supremum does not exist in the set), but when a maximum exists it is also the supremum.
        \textbf{Lemma} \textit{Assume $s \in \textbf{R}$ is an upper bound for a set $A \in \textbf{R}$. Then, $s = \text{sup } A$ if and only if, for every choice of $\epsilon > 0$, there exists an element $a \in A$ satisfying $s - \epsilon < a$}
        \begin{proof}
            Given that $s$ is an upper bound, $s$ is the leastupper bound if and only if any number smaller than $s$ is not an upper bound.
            \newline \indent $\Rightarrow$ Assume $s = \text{sup } A$ and consider $s - \epsilon$, where $\epsilon > 0$ has been chosen. Since $s - \epsilon < s$, $s - \epsilon$ is not an upper bound for \textit{A}. So there must be an $a \in A$ such that $s - \epsilon < a$.
            \newline \indent $\Leftarrow$ Assume s is an upper bound so that for every $\epsilon > 0$, $s - \epsilon$ is no longer an upper bound for A. $s = \text{sup } A$ since $s$ is an upper bound, and any real number $b < s$ is not an upper bound. This is apparent by setting $\epsilon = s - b$. 
        \end{proof}