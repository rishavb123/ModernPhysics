\section{Work, Energy, and the Conservation of Energy}
    For an object traveling in one dimension with a constant net force $F$ and with velocity changing linearly with time ($v \propto t$), we have
    \begin{equation*}
        F \cross (x_f - x_i) = \frac{1}{2}mv_f^2 - \frac{1}{2}mv_i^2
    \end{equation*}
    The quantity $mv^2 / 2$ is called the \textbf{kinetic energy} $K$, while the left side is referred to as the \textbf{work} $W$ done by the force. For forces that are not constant we have
    \begin{equation*}
        W = \int_{x_i}^{x_f}F(x) dx
    \end{equation*}
    To extend to multiple dimensions
    \begin{equation*}
        W = \int_{\va{r}_i}^{\va{r}_f} \va{F} \vdot d\va{r}
    \end{equation*}
    With these new definitions, the first equation of this section can be written as
    \begin{equation*}
        W_{net} = \Delta K = K_f - K_i
    \end{equation*}
    aka the \textbf{work-energy theorem}.