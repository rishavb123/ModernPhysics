\section{Discussion: the Irrationality of $\sqrt{2}$}
    \begin{theorem} 
        There is no rational number whose square is 2.
    \end{theorem}
    \begin{proof}
        A rational number can be written in the form $\frac{p}{q}$ where p and q are integers. We will use an indirect proof. First, assume there is a rational so that its square is 2. It can be written that
        \begin{equation*}
            (\frac{p}{q})^2 = 2
        \end{equation*}
        We can assume $p$ and $q$ have no common factors since they would cancel anyways and give us a new $p$ and $q$. Now we can written
        \begin{equation*}
            p ^ 2 = 2q^2
        \end{equation*}
        which implies that $p^2$ is an even number, which implies $p$ is an even number. So we can let $p = 2r$. Plugging this in
        \begin{equation*}
            2r^2 = q^2
        \end{equation*}
        With the same logic as for with $p$, $q$ is also even. So $p$ and $q$ share a common factor of 2 which contradicts the assumption made in the beginning that they share no common factors.
    \end{proof}
    \subsection*{Important number systems as sets}
        \textit{Natural Numbers}
        \begin{equation*}
            \textbf{N} = \{1,2,3,4,5,\dots\}
        \end{equation*}
        \indent Addition works well he, but there is no additive identity or inverse.
        \newline \newline
        \textit{Integers}
        \begin{equation*}
            \textbf{Z} = \{\dots,-2,-1,0,1,2,\dots\}
        \end{equation*}
        \indent This includes the additive identity (0) and the additive inverses, which define subtraction. The multiplicative identity is 1, but for multiplicative inverses we need to extend to \dots
        \newline \newline
        \textit{Rational Numbers}
        \begin{equation*}
            \textbf{Q} = \{\text{all fractions $\frac{p}{q}$ where $p$ and $q$ are integers and $q$ $\neq$ 0}\}
        \end{equation*}
        \indent The multiplicative inverses define division. All of these properties of \textbf{Q} make it into a \textit{field}. A field is any set where addition and multiplication are well-defined operations that are commutative, associative, and obey the distributive property: $a(b + c) = ab + bc$. There must be an additive and multiplicative identity, and each element must have an additive and multiplicative inverse.
        \newline
        \indent The set \textbf{Q} has a natural \textit{order}. Given two rational numbers $r$ and $s$, one of the following is true:
        \begin{equation*}
            \text{$r < s$, $r = s$, or $r > s$}
        \end{equation*}
        This ordering is transitive: if $r < s$ and $s < t$, then $r < t$. Also, between any two rational numbers, $r$ and $s$, there is a rational number between them: $\frac{r+s}{2}$, which implies that rational numbers are densely packed. 
        \newline 
        \indent \textbf{Q} is has holes in the spots of irrationals, such as $\sqrt{2}$ and $\sqrt{3}$. To fill these we add \dots
        \newline \newline
        \textit{Real Numbers}
        \begin{equation*}
            \textbf{R} = \{\text{all real numbers}\}
        \end{equation*}
        \indent Just like \textbf{Q}, \textbf{R} is a field. \textbf{R} is added as a superset of \textbf{Q}. \textbf{N} $\subseteq$ \textbf{Z} $\subseteq$ \textbf{Q} $\subseteq$ \textbf{R}.