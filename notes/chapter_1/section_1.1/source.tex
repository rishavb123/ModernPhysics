\section{Newton's Laws}
\indent Isaac Newton introduced the three basic laws of mechanics that are known as Newton's Laws.
\subsection*{Newton's Second Law}:
\begin{equation}
    \va{F}_{net} = m\va{a}
\end{equation}
Sometimes this equation is written as
\begin{equation}
    \va{F}_{net} = \frac{d\va{p}}{dt}
\end{equation}
to account for a changing mass, where $\va{p} = m\va{v}$. 
\subsection*{Newton's First Law}
Newton's first law is a special case of the second law:
\begin{equation}
    \text{If $\va{F}_{net} = \va{0}$, the motion is uniform}
\end{equation}
which means that velocity is constant and accusation is 0.
\subsection*{Gravity}
Acceleration caused by gravity is the same on every object. This seems contradictory to what we have seen from Newton's second law, $\va{a} = \va{F}_{new} / m$, but since the force of gravity is proportional to the mass, the acceleration is independent of the mass.
\subsection*{Hooke's Law}
The equation for spring force is $F = -kx$. So,
\begin{equation}
    -kx = m\frac{d^2x}{dt}
\end{equation}
The solution of this differential equation is 
\begin{equation}
    x(t) = A\sin(\omega t + \varphi)
\end{equation}
where $A$ is the \textit{amplitude} of the motion, and $\varphi$ is the textit{phase}. Both of these quantities are determined by the initial conditions. $\omega$ is determined by the spring and the mass:
\begin{equation}
    \omega = \sqrt{k / m}
\end{equation}
where $\omega$ is the \textit{angular frequency}, which is related to the \textit{period} $T$ and \textit{frequency} $f$. 
\begin{equation}
    T = \frac{1}{f}
\end{equation}
so $f = \omega/2\pi$.
\subsection*{Newton's Third Law}
Every force has an equal and opposite reaction force.
\begin{equation}
    \va{F}_12 = -\va{F}_21
\end{equation}
This can be restated in terms of momentum.
\begin{equation}
    \frac{d\va{p_1}}{dt} = -\frac{d\va{p_2}}{dt} \text{ or } \frac{d}{dt} (\va{p_1} + \va{p_2}) = 0
\end{equation}
So,
\begin{equation}
    \va{P}_{tot} = \text{a constant vector}
\end{equation}