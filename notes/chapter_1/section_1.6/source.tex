\section{The Atomic Structure of Matter}
    Bulk matter comes in three states: gaseous, liquid, and solid. Each of these correspond to different organizations of atoms or molecules that make up matter. In gas, particles are spread apart; in liquid, the atoms are closely bunched; in solid, the atoms are closely bunched and ordered in their arrangement. Electromagnetic forces caused by the presence of \textbf{electric charge} cause these particles to be grouped differently. Charge can either be negative or positive, and like charges repel while different charges attract. The force also depends on the distance between particles.
    \newline \indent
    Rutherford's experiment consisted of firing \textbf{alpha particles} (+$2e$), aka helium nuclei, at a thin foil. Most of the particles went straight through but a few deflected off the nucleus (when it passes at a distance R away). It feels a force due to the positive charge $Ze$:
    \begin{equation*}
        F = \frac{(2e)(Ze)}{4\pi\epsilon_0R^2}
    \end{equation*}
    We can estimate the deflection angle (since it is small $\theta \approx \sin\theta \approx \tan\theta$. The time the particle spends near the charge is approximately $2R/v$
    \begin{equation*}
        \frac{\Delta p}{p} = \frac{F\Delta t}{M_\alpha v}=\frac{F(2R/v)}{M_\alpha v}=\frac{4Ze^2/4\pi\epsilon_0R}{M_\alpha v^2}
    \end{equation*}
    Rutherford's explanation for the small number of deflections was that the radius of the positive charge distribution is $10^4$ times smaller than the $10^{-10}$ m from the previous model. The atom was still $10^{-10}$ m, but most of it is electron orbitals while the nucleus only takes up a small portion, which orbit like planets since the coulomb force varies as $1/r^2$ just like gravity.